\PassOptionsToPackage{unicode=true}{hyperref} % options for packages loaded elsewhere
\PassOptionsToPackage{hyphens}{url}
%
\documentclass[]{article}
\usepackage{lmodern}
\usepackage{amssymb,amsmath}
\usepackage{ifxetex,ifluatex}
\usepackage{fixltx2e} % provides \textsubscript
\ifnum 0\ifxetex 1\fi\ifluatex 1\fi=0 % if pdftex
  \usepackage[T1]{fontenc}
  \usepackage[utf8]{inputenc}
  \usepackage{textcomp} % provides euro and other symbols
\else % if luatex or xelatex
  \usepackage{unicode-math}
  \defaultfontfeatures{Ligatures=TeX,Scale=MatchLowercase}
    \setmainfont[]{Verdana}
\fi
% use upquote if available, for straight quotes in verbatim environments
\IfFileExists{upquote.sty}{\usepackage{upquote}}{}
% use microtype if available
\IfFileExists{microtype.sty}{%
\usepackage[]{microtype}
\UseMicrotypeSet[protrusion]{basicmath} % disable protrusion for tt fonts
}{}
\IfFileExists{parskip.sty}{%
\usepackage{parskip}
}{% else
\setlength{\parindent}{0pt}
\setlength{\parskip}{6pt plus 2pt minus 1pt}
}
\usepackage{hyperref}
\hypersetup{
            pdftitle={Preregistration - The effect of dominance rank on female reproductive success in social mammals},
            pdfauthor={Shivani1,2; Elise Huchard3; Dieter Lukas2*},
            pdfborder={0 0 0},
            breaklinks=true}
\urlstyle{same}  % don't use monospace font for urls
\usepackage[margin=1in]{geometry}
\usepackage{graphicx,grffile}
\makeatletter
\def\maxwidth{\ifdim\Gin@nat@width>\linewidth\linewidth\else\Gin@nat@width\fi}
\def\maxheight{\ifdim\Gin@nat@height>\textheight\textheight\else\Gin@nat@height\fi}
\makeatother
% Scale images if necessary, so that they will not overflow the page
% margins by default, and it is still possible to overwrite the defaults
% using explicit options in \includegraphics[width, height, ...]{}
\setkeys{Gin}{width=\maxwidth,height=\maxheight,keepaspectratio}
\setlength{\emergencystretch}{3em}  % prevent overfull lines
\providecommand{\tightlist}{%
  \setlength{\itemsep}{0pt}\setlength{\parskip}{0pt}}
\setcounter{secnumdepth}{0}
% Redefines (sub)paragraphs to behave more like sections
\ifx\paragraph\undefined\else
\let\oldparagraph\paragraph
\renewcommand{\paragraph}[1]{\oldparagraph{#1}\mbox{}}
\fi
\ifx\subparagraph\undefined\else
\let\oldsubparagraph\subparagraph
\renewcommand{\subparagraph}[1]{\oldsubparagraph{#1}\mbox{}}
\fi

% set default figure placement to htbp
\makeatletter
\def\fps@figure{htbp}
\makeatother


\title{Preregistration - The effect of dominance rank on female reproductive
success in social mammals}
\author{Shivani\textsuperscript{1,2} \and Elise Huchard\textsuperscript{3} \and Dieter Lukas\textsuperscript{2}*}
\date{2020-06-12}

\begin{document}
\maketitle

\hypertarget{affiliations}{%
\subparagraph{Affiliations:}\label{affiliations}}

\begin{enumerate}
\def\labelenumi{\arabic{enumi}.}
\tightlist
\item
  Indian Institute of Science Education and Research Kolkata
\item
  Department of Human Behavior, Ecology \& Culture; Max Planck Institute
  for Evolutionary Anthropology Leipzig
\item
  Institut des Sciences de l'Evolution de Montpellier, Centre National
  de la Recherche Scientifique, Université de Montpellier
\end{enumerate}

Correspondence: *
\href{mailto:dieter_lukas@eva.mpg.de}{\nolinkurl{dieter\_lukas@eva.mpg.de}}

~

\hypertarget{abstract}{%
\subsubsection{ABSTRACT}\label{abstract}}

Life in social groups, while potentially providing social benefits,
inevitably leads to conflict among group members. In many social
mammals, such conflicts lead to the formation of dominance hierarchies,
where high-ranking individuals consistently outcompete other group
members. Given that competition is a fundamental tenet of the theory of
natural selection, it is generally assumed that high-ranking individuals
have higher reproductive success than lower-ranking individuals.
Previous reviews have indicated large variation across populations on
the potential effect of dominance rank on reproductive success in female
mammals. Here, we propose to perform a meta-analysis based on 444 effect
sizes from 187 studies on 86 mammal species to determine whether (1)
dominance rank is generally positively associated with reproductive
success and whether the approach different studies have taken to answer
this question influences the strength of the effect, (2) whether
life-history mechanisms might mediate the relationship between rank and
reproductive success, (3) whether high-ranking females are more likely
to have higher success when resources are limited, and (4) whether the
social environment might mitigate rank differences on reproductive
success. This preregistration lays out the background, objective,
predictions we will test, and proposed methods for our study.

~

\hypertarget{a.-preregistration-state-of-the-data}{%
\subsubsection{A. PREREGISTRATION: STATE OF THE
DATA}\label{a.-preregistration-state-of-the-data}}

The literature search was completed before the first submission of the
preregistration. All variables that were coded directly from the source
publications (Z transformed effect size, variance, sample size, species
identity, aspect of reproductive success, classification of rank,
duration of study, population type, and social group size) were also
entered prior to the first submission. None of the potential explanatory
variables have been entered. No analyses, including any descriptive
analyses or visualisations of the complete dataset, were performed prior
to submission of the preregistration. In July 2019, S worked with a
preliminary subset of the data (143 effect sizes), and investigated
publication bias, the overall mean and variance in effect sizes, and
whether effect sizes differed according to which reproductive output was
measured.

~

\hypertarget{b.-background}{%
\subsubsection{B. BACKGROUND}\label{b.-background}}

In order for social groups to persist, group members need to find
strategies to deal with the conflicts that inevitably occur (Ward and
Webster (2016)). In many female social mammals, conflicts and aggressive
interactions are associated with the formation of different types of
hierarchies. In singular cooperative breeders, a single dominant
breeding female suppresses reproduction in subordinate group members,
who rarely fight amongst each other until an opportunity to become
dominant opens (Solomon, French, and others (1997)). In many species
where multiple breeding females form stable groups, females can be
arranged in stable linear hierarchies, where mothers help their
daughters to inherit their rank in their matriline (Holekamp and Smale
(1991)). In another set of species, hierarchies are more flexible as a
female's rank depends on her body size, condition, or availability of
coalition partners (Pusey (2012)). Given that, in species in which
dominance hierarchies structure social groups, females can always be
attributed either a low or a high rank, it has remained unclear whether
and when there is selection on females to compete for a high rank or
whether selection is on finding a place in the hierarchy.

The prevailing assumption is that high ranking females benefit from
their dominant status because outcompeting other females is expected to
provide them with priority of access to resources (Ellis (1995), Pusey
(2012)). Subordinates are expected to accept their status, because
despite having lower reproductive success than dominants, they have few
outside options and would presumably face high costs, or have even lower
success if they tried to challenge for the dominant status or to
reproduce independently (Alexander (1974), Vehrencamp (1983)). An
alternative assumption however is that both dominants and subordinates
gain from arranging themselves in a hierarchy to avoid the overt
fighting that occurs whenever differentially aggressive individuals
repeatedly interact (West (1967)). All individuals make a compromise,
such that they all balance the potential benefits of their respective
positions with the potential costs (Williams (1966)).

Previous reviews have found that while high ranking female mammals
frequently appear to have higher reproductive success, there are many
populations where such an association has not been found (Pusey (2012),
Clutton-Brock and Huchard (2013)). Most studies that brought together
the evidence have focused on primates and generally only provided
qualitative summaries of the evidence (Fedigan (1983), Ellis (1995),
Stockley and Bro-Jørgensen (2011)). One meta-analysis across primates
investigated whether life history might mediate the strength of the
association between dominance and reproductive success and found that
high-ranking females had higher fecundity benefits in species with a
longer lifespan (Majolo et al. (2012)). However, there is no systematic
assessment of the many potential factors that have been suggested to
mitigate the relationship between rank and reproductive success when
high rank might not be associated with higher reproductive success.

~

\hypertarget{c.-objective}{%
\subsubsection{C. OBJECTIVE}\label{c.-objective}}

In this study, we will perform a quantitative assessment of the strength
of the relationship between dominance rank and reproductive success in
female social mammals and explore factors that might mediate this
relationship. Our objective is to identify the sources and ranges of
variation in the relationship between rank and reproductive success and
predict that the relationship will be influenced by differences in
life-history, ecology, and sociality. We address our objective through
the following questions, by testing the corresponding predictions:

\hypertarget{does-high-rank-generally-lead-to-higher-reproductive-success-for-females-in-social-mammals}{%
\paragraph{\texorpdfstring{\textbf{1) Does high rank generally lead to
higher reproductive success for females in social
mammals?}}{1) Does high rank generally lead to higher reproductive success for females in social mammals?}}\label{does-high-rank-generally-lead-to-higher-reproductive-success-for-females-in-social-mammals}}

We expect that, overall, high dominance rank has a positive effect on
reproductive success.

\hypertarget{what-are-the-life-history-traits-that-mediate-the-benefits-of-rank-on-reproductive-success}{%
\paragraph{\texorpdfstring{\textbf{2) What are the life history traits
that mediate the benefits of rank on reproductive
success?}}{2) What are the life history traits that mediate the benefits of rank on reproductive success?}}\label{what-are-the-life-history-traits-that-mediate-the-benefits-of-rank-on-reproductive-success}}

We expect that dominants have higher reproductive success predominantly
in species in which females have the ability to quickly produce large
numbers of offspring.

\hypertarget{what-are-the-ecological-conditions-that-mediate-the-benefits-of-rank-on-reproductive-success}{%
\paragraph{\texorpdfstring{\textbf{3) What are the ecological conditions
that mediate the benefits of rank on reproductive
success?}}{3) What are the ecological conditions that mediate the benefits of rank on reproductive success?}}\label{what-are-the-ecological-conditions-that-mediate-the-benefits-of-rank-on-reproductive-success}}

We expect that differences in reproductive potential will be
particularly marked if resources are limited and monopolizable.

\hypertarget{what-are-the-social-circumstances-that-mediate-the-benefits-of-rank}{%
\paragraph{\texorpdfstring{\textbf{4) What are the social circumstances
that mediate the benefits of
rank?}}{4) What are the social circumstances that mediate the benefits of rank?}}\label{what-are-the-social-circumstances-that-mediate-the-benefits-of-rank}}

We expect that the association between dominance rank and reproduction
is stronger in species living in more stable and structured social
groups.

~

\hypertarget{d.-predictions}{%
\subsubsection{D. PREDICTIONS}\label{d.-predictions}}

To answer these questions, we will assess the following predictions. All
our predictions consider the potential direct influence of a specific
variable on the size of the effect of dominance rank on reproductive
success. The predictions present the direction of the influence we
consider a-priori most likely. We will report all results, but in
instances where influences are opposite to what we predict further
studies will be necessary to place these results in context. In
addition, several of the variables we will include are likely to
influence each other. Accordingly, analyses with single variables might
not necessarily show the predicted direct influence even if it is
present (e.g.~there might not be a positive relationship between a
social system and the size of the effects if species with this
particular social system primarily occur in environments where the size
of the effect is expected to be smaller). While deciphering all the
potential relationships among the variables we include is beyond the
scope of this study, we will also perform analyses accounting for these
potential interactions among variables by performing path analyses. We
focus on instances where we expect that one variable might remove or
change the direction of the influence of another variable, and present
these at the end of the predictions. ~

\hypertarget{does-high-rank-generally-lead-to-higher-reproductive-success-for-females-in-social-mammals-1}{%
\paragraph{\texorpdfstring{\textbf{1) Does high rank generally lead to
higher reproductive success for females in social
mammals?}}{1) Does high rank generally lead to higher reproductive success for females in social mammals?}}\label{does-high-rank-generally-lead-to-higher-reproductive-success-for-females-in-social-mammals-1}}

\emph{P1.1: Publication bias does not influence our sample of effect
sizes.}

We do not predict a publication bias but that our sample will include
studies showing small effect sizes with small sample sizes. Most studies
set out to test if high dominance might lead to both benefits and costs,
and previous meta-analyses did not detect signals of publication bias
(e.g. Majolo et al. (2012)).

\emph{P1.2: Overall, high dominance rank will be associated with higher
reproductive success.}

We predict that, taking into account the power of the different studies,
the combined effect of high rank on reproductive success will be
positive. Previous studies that summarized existing evidence (e.g.
Majolo et al. (2012), Pusey (2012)) found support for the consensual
framework in socio-ecology which argues that high ranking females
generally have higher reproductive success than low ranking females.

\emph{P1.3 Effect sizes from the same population and the same species
will be similar.}

We predict that studies that have been conducted on the same species,
and in particular at the same site, will report similar effects of
dominance rank on reproductive success. For some long-term studies,
multiple studies have been performed using slightly different methods
and/or data from different years which might include the same set of
individuals leading to very similar effect size estimates. For studies
of the same species from different sites, we expect similarities because
many aspects of the life-history and social system that will shape the
relationship between rank and reproductive success will be conserved.

\emph{P1.4: Closely related species will show similar effects of
dominance rank on reproductive success.}

We predict that effect sizes of the relationship between dominance rank
and reproductive success will be more similar among closely related
species (Chamberlain et al. (2012)) because methodological approaches
can be specific to specific Orders (e.g.~ungulates are studied
differently than primates) and because closely related species share
life history, social and ecological traits that might shape the
influence of rank on reproductive success.

\emph{P1.5: Effect sizes depend on the approach used.}

We expect that some of the variation in effect size across studies
arises from methodological differences:

\begin{enumerate}
\def\labelenumi{(\roman{enumi})}
\tightlist
\item
  we predict lower effect sizes for studies of captive populations
  compared to wild populations: while the absence of stochastic events
  in captivity might mean that dominance is more consistently associated
  with certain benefits, the effects of high dominance rank on
  reproductive success will be reduced because of lower competition over
  resources;
\item
  we predict lower effect sizes for studies where rank was measured
  based on agonistic interactions rather than on size or age because
  size and age are frequently directly associated with differences in
  female reproduction and clear differences between dominants and
  subordinates may indicate the existence of castes that tend to be
  associated with strong reproductive monopolization (Lukas and
  Clutton-Brock (2018)); and
\item
  we predict different effect sizes for studies classifying individuals
  into two or three rank categories compared to linear ranking depending
  on the social system. In cases where there is usually a single
  dominant female (singular cooperative breeders, such as meerkats),
  using a linear regression between each individuals' rank and its
  reproductive success will likely estimate a lower effect size because
  such an approach assumes differences in rank or reproductive success
  among the subordinates when there are none. In contrast, grouping
  individuals into categories to compare dominants to subordinates will
  capture actual differences more accurately. In cases where several
  females breed (plural breeders, such as hyenas) and are ordered in a
  linear hierarchy, a linear regression will exploit the full
  information available on individual differences in rank and
  reproductive success, whereas grouping individuals will lead to a loss
  of resolution, at a risk of underestimating the differences between
  highest and lowest ranking individuals. We performed simulations to
  determine the extent to which this choice of approach skews the effect
  sizes and found that it can lead to differences of more than 35\%
  between the true and the estimated effect sizes. For illustration, we
  include this simulation in our code.
\end{enumerate}

~

\hypertarget{what-are-the-life-history-traits-that-mediate-the-benefits-of-rank-on-reproductive-success-1}{%
\paragraph{\texorpdfstring{\textbf{2) What are the life history traits
that mediate the benefits of rank on reproductive
success?}}{2) What are the life history traits that mediate the benefits of rank on reproductive success?}}\label{what-are-the-life-history-traits-that-mediate-the-benefits-of-rank-on-reproductive-success-1}}

\emph{P2.1: High dominance rank will benefit females more than their
offspring.}

We predict that high rank is more likely to be associated with higher
reproductive success in studies that measured female age at first
reproduction, number of offspring born per year or across a lifetime, or
female survival rather than the survival of their offspring. While in
cooperatively breeding species reproductive suppression might impact
offspring survival, in plural breeders offspring survival is more likely
to be influenced by factors that are outside of the control of females,
such as infanticide by new males (Cheney et al. (2004)).

\emph{P2.2: Dominance will have stronger effects on immediate
reproductive success in species in which females produce many offspring
over a short time period.}

One key mechanism that has been proposed is that females with high
dominance rank have priority of access to resources during periods when
these resources are limited, which in turn can increase their
reproductive success. Accordingly, we predict stronger effects of rank
on measures of immediate reproductive success (offspring production,
offspring survival) in species in which females have higher energetic
investment into reproduction, with larger litter sizes and shorter
interbirth intervals (Lukas and Huchard (2019)). In contrast, in
long-lived species in which females produce only single offspring at
long intervals, high-ranking females are expected to have less
opportunity to translate short-term resource access into immediate
reproductive success but might store energy to potentially increase
their own survival or lifetime reproductive success.

~

\hypertarget{what-are-the-ecological-conditions-that-mediate-the-benefits-of-rank-on-reproductive-success-1}{%
\paragraph{\texorpdfstring{\textbf{3) What are the ecological conditions
that mediate the benefits of rank on reproductive
success?}}{3) What are the ecological conditions that mediate the benefits of rank on reproductive success?}}\label{what-are-the-ecological-conditions-that-mediate-the-benefits-of-rank-on-reproductive-success-1}}

\emph{P3.1: Positive effects of high dominance rank on reproductive
success will be stronger in populations in which females feed on
resources that are more monopolizable.}

We predict that high rank will have stronger effects on reproductive
success in fruit- and meat-eaters compared to herbivores or omnivores.
One of the main expected benefits of high rank is priority of access to
resources, which should be more relevant in populations in which
resources can be monopolized (Fedigan (1983)).

\emph{P3.2: Effects of dominance rank on reproductive success will be
more pronounced in populations living in harsh environments.}

We predict that the effect of rank on reproductive success will be
stronger in populations in which resources are limited because they live
in harsh and unpredictable environments. Previous studies have shown
that cooperatively breeding species are more likely to occur in such
environments (Lukas and Clutton-Brock (2017)), but we also expect
stronger effects among plural breeding populations living in harsh
environments.

\emph{P3.3: Effects of dominance rank on reproductive success will be
more pronounced in populations with high densities of individuals.}

We predict that the effect of rank on reproductive success will be
stronger in populations in which more individuals share a limited amount
of space. At higher population densities, social groupings and
interactions are more likely and competition over resources is expected
to be stronger.

~

\hypertarget{what-are-the-social-circumstances-that-mediate-the-benefits-of-rank-1}{%
\paragraph{\texorpdfstring{\textbf{4) What are the social circumstances
that mediate the benefits of
rank?}}{4) What are the social circumstances that mediate the benefits of rank?}}\label{what-are-the-social-circumstances-that-mediate-the-benefits-of-rank-1}}

\emph{P4.1: Benefits of rank will be most pronounced in cooperatively
breeding species.}

We predict that rank effects on reproduction will be higher in
cooperative breeders, where the dominant female is often the only
breeding female because she suppresses the reproduction of subordinate
females (Digby, Ferrari, and Saltzman (2006)), compared to plural
breeders, where aggressive behaviour is more targeted and limited to
access over specific resources.

\emph{P4.2: For plural-breeders, the time-scales at which the
reproductive benefits of dominance accrue depend on how individuals
achieve high rank.}

We predict that in populations of plural breeders in which groups
contain multiple breeding females, the way in which these females
compete over dominance will influence the potential benefits of high
rank. In populations in which female rank depends primarily on age, high
ranking females will have higher reproductive success for short periods
of time because changes in rank are expected to occur regularly, and
because high rank may only be reached towards the end of their
reproductive life (Thouless and Guinness (1986)). In societies in which
female rank depends primarily on size or condition, rank effects on
reproductive success are expected to be expressed on intermediate time
frames, as individuals may not be able to maintain a larger relative
size or condition over lifetime but they are expected to acquire rank
relatively early in their reproductive life (Giles et al. (2015),
Huchard et al. (2016)). In societies in which female rank primarily
depends on nepotism, and ranks are often inherited and stable across a
female's lifetime, we predict that effects of rank on reproductive
success will be strongest when measured over long periods because small
benefits might add up to substantial differences among females (Frank
(1986)) whereas stochastic events might reduce differences between
females on shorter time scales (Cheney et al. (2004)).

\emph{P4.3: Dominance rank will have stronger effects on reproductive
success in populations in which females are philopatric in comparison to
populations where females disperse to breed.}

We predict that effects of rank on reproductive success will be lower in
populations in which adult females are able to leave their group and
join other groups compared to populations in which females cannot breed
outside their natal group. In populations in which females are
philopatric, they are likely to have support from female kin which can
strengthen dominance differences (Lukas and Clutton-Brock (2018)). In
addition, in species where females can change group membership easily,
females are expected to join those groups where they have the best
breeding option available to them (Vehrencamp (1983)).

\emph{P4.4: In plural breeding species, dominance will have stronger
effects on reproductive success when the number of females in the group
is smaller.}

We predict that the effect of rank on reproductive success will be
stronger in plural breeding populations in which there are fewer females
per group, because dominant females will be more likely to interfere in
reproductive attempts when there are fewer subordinates (Clutton-Brock
et al. (2010) and because increased competition in larger groups is
expected to reduce reproductive success even among dominants (Van
Noordwijk and Van Schaik (1988)).

\emph{P4.5 Dominance rank will be more strongly associated with
reproductive success in populations in which average relatedness among
female group members is high.}

We predict that the relationship between dominance rank and reproductive
success will be more pronounced in species in which social groups
primarily consist of close kin compared to groups composed of unrelated
females. Groups with high levels of average kinship among females are
those where groups are small, females remain philopatric (Lukas et al.
(2005)), and females have support to establish their positions (Lukas
and Clutton-Brock (2018)), which all are expected to lead to higher
benefits of high rank.

\emph{P4.6 Dominance rank will be more strongly associated with
reproductive success in populations in which variance in relatedness
among female group members is high.}

In addition to levels of average relatedness among group females, we
also predict that the relationship between dominance rank and
reproductive success will be more pronounced in species in which there
is high variance in relatedness, with females being closely related to
some group members but not to others, as compared to species in which
group females are either all related or all unrelated. In several
species with female philopatry, groups are structured into matrilines
(Fortunato (2019)). Members of the same matriline tend to support each
other in interactions with unrelated females, likely reinforcing
differences among females.

\emph{P4.7 The effect of dominance on reproductive success will be less
pronounced in populations in which females regularly form coalitions.}

We predict that high ranking females will have less pronounced
reproductive benefits in species in which females form strategic
coalitions with others (Bercovitch (1991)). Individuals have been
suggested to form strategic coalitions to level the reproduction of
others (Pandit and Schaik (2003)) and these coalitions are less likely
in cooperatively breeding species (Lukas and Clutton-Brock (2018)).

\emph{P4.8 Dominance rank will have less effect on reproductive success
in populations in which there is intense inter-sexual conflict.}

We predict that the association between high dominance rank and
increased reproductive success of females will be lower in populations
in which males compete intensively over reproductive opportunites
because this leads to intersexual conflict that harms females. In such
populations, males tend to be aggressive towards females and males
taking up tenure in a group tend to kill offspring indiscriminately or
might even target offspring of high-ranking females (Fedigan and Jack
(2013)), reducing any potential differences between high- and
low-ranking females. We will assess whether high ranking females benefit
less from their positions in populations in which groups show strong
female-biased sex composition, or in which males regularly commit
infanticide, or with strong sexual size dimorphism with males being much
larger than females.

~

\hypertarget{potential-interactions-among-predictor-variables}{%
\paragraph{\texorpdfstring{\textbf{Potential interactions among
predictor
variables}}{Potential interactions among predictor variables}}\label{potential-interactions-among-predictor-variables}}

\emph{Studies performed on wild versus captive individuals and using
different measures of reproductive success might not only differ in the
overall strength of the effect of rank on reproductive success, but also
in how other variables influence this effect.}

\emph{Higher population density {[}predicted to lead to larger effect
sizes{]} might be associated with larger group sizes {[}smaller effect
sizes predicted{]}, leading to an interactive influence on the strength
of the effect sizes of dominance rank on reproductive success.}

\emph{Smaller group sizes {[}larger effect sizes predicted) might be
associated with more intense intersexual conflict {[}smaller effect
sizes predicted{]}, leading to an interactive influence on the strength
of the effect sizes of dominance rank on reproductive success.}

\emph{Monopolizable resources {[}larger effect sizes predicted{]} might
be associated with reduced population density {[}smaller effect sizes
predicted{]}), leading to an interactive influence on the strength of
the effect sizes of dominance rank on reproductive success.}

\emph{Environmental harshness {[}larger effect sizes predicted{]} might
be associated with reduced population density {[}smaller effect sizes
predicted{]}), leading to an interactive influence on the strength of
the effect sizes of dominance rank on reproductive success.}

\emph{Female philopatry {[}larger effect sizes predicted{]} might be
associated with increased group sizes {[}smaller effect sizes
predicted{]}), leading to an interactive influence on the strength of
the effect sizes of dominance rank on reproductive success.}

~

\hypertarget{e.-methods}{%
\subsubsection{E. METHODS}\label{e.-methods}}

\hypertarget{literature-search}{%
\paragraph{\texorpdfstring{\textbf{Literature
search}}{Literature search}}\label{literature-search}}

The literature search was performed by S \& DL. We started with the
references in the previous major reviews and meta-analyses on the
association between dominance and reproduction in female mammals (see
below for inclusion criteria): Fedigan (1983) (8 studies on female
primates entered), Ellis (1995) (16 studies entered / 5 studies not
entered on female non-primates, 38 studies entered / 22 studies not
entered on female primates), Brown and Silk (2002) (28 studies entered /
7 studies not entered on female primates), Stockley and Bro-Jørgensen
(2011) (12 studies entered / 2 studies not entered on female
non-primates, 11 studies entered / 1 study not entered on female
primates), Majolo et al. (2012) (26 studies entered / 2 studies not
entered on female primates), Pusey (2012) (45 studies entered / 2
studies not entered on female primates), and Clutton-Brock and Huchard
(2013) (8 studies entered / 1 study not entered on female primates, 6
studies entered / 1 study not entered on female non-primates). Next, we
performed database searches in Google Scholar and Pubmed, first by
identifying articles citing these major reviews and next by searching
with the terms ``dominance, reproductive success/reproduction, female,
mammal'', and ``rank, reproductive success/reproduction, female,
mammal'', ``sex ratio, dominance, female, mammal'' (searches performed
July 2019-January 2020). We limited our checks to the first 1000 results
for all searches.

We checked the titles and abstracts to identify studies that observed
dominance interactions and reproductive success in social groups of
interacting female non-human mammals. We selected studies that measured
the association between dominance rank and at least one aspect of female
reproductive success and reported the data or a test-statistic. For both
dominance and reproductive success, we only included studies that had
direct measures, not secondary indicators. For dominance, we excluded
studies where authors did not explicitly determine dominance
relationships and only assumed that traits such as size, presence in
core areas, or reproductive success itself indicate dominance. We did
however include studies where authors established dominance hierarchies,
found that they are associated with some other trait such as size or
condition, and subsequently used the other trait to measure dominance.
For reproductive success, we excluded studies that measured traits such
as mating frequency or access to food resources which were assumed but
not known to influence reproductive success (excluding studies that:
measured the size of individuals to argue about dominance; assumed that
females in core areas are dominant; assigned dominance to females based
on how successful they are; recorded mating success not reproductive
success; linked dominance to behaviour assumed to potentially link to
reproductive success). We included all kinds of academic publications,
from primary articles published in peer-reviewed journals through
reviews, books and book chapters, and unpublished PhD theses.

\hypertarget{variables-their-definitions-and-their-sources}{%
\paragraph{\texorpdfstring{\textbf{Variables, their definitions, and
their
sources}}{Variables, their definitions, and their sources}}\label{variables-their-definitions-and-their-sources}}

\textbf{Variables coded directly from the relevant publications:} All
data from the literature search on publications reporting the effect of
dominance rank on reproductive success has been entered prior to the
first submission of the preregistration. S and DL performed the data
extraction. We initially coded eight papers independently, for which we
both extracted the same values and classified the approaches in the same
way. We extracted the relevant information to calculate the effect size
and its associated variance. In addition, we coded a set of variables to
characterize the methodological approach. The dataset contains 444
effect sizes from 187 studies on 86 mammalian species. A copy of the
\href{https://github.com/dieterlukas/FemaleDominanceReproduction_MetaAnalysis}{datafile
is available here}

\textbf{Z-transformed effect size}: we converted all effect sizes to
Z-transformed correlation coefficients (Zr). In cases where articles
reported a pairwise correlation coefficient, we directly use this value.
In cases where authors had used alternative statistical approaches
(e.g.~t-test comparison between two groups of individuals), the test
statistics were converted to the statistic `r' using formulas provided
by Lakens (2013), Lajeunesse et al. (2013), and Wilson (2019). In cases
where authors reported individual-level data reflecting dominance rank
and reproductive success (for example in the form of a table that listed
for groups of dominants and subordinates their mean and deviation of
reproductive success or for every individual their rank and reproductive
success), we calculated correlation coefficients directly from a 2-by-2
frequency table (when comparing classes of high- to low-ranking
individuals) or from linear regressions (when individuals had continuous
ranks). In cases where studies simply stated that ``all dominants bred
but none of the subordinates'' we assumed an error of 0.5\% for both
dominants not breeding and subordinates breeding to obtain the sampling
variance estimates. We extracted separate effect sizes for each reported
analysis: for example, if authors reported separately associations
between dominance rank and mortality of offspring to 1 year and to
independence, we obtained two effect sizes from this population
reflecting infant survival. We Z-transformed all correlation
coefficients to control for the asymptotic distribution of these values.
We changed the sign of the effect sizes to make them consistent across
studies. This was necessary because dominance rank was coded differently
across studies, for example sometimes studies assigned dominant
individuals the lowest value by starting a count from 1, whereas in
other cases they were assigned the highest value to reflect the
proportion of other females they are dominant over. We set the sign of
effect sizes such that positive values mean that higher ranking
individuals have shorter interbirth intervals, higher survival as adults
and of their infants, higher infant production (e.g.~larger litter
sizes, higher probability of breeding), and higher lifetime reproductive
success (e.g.~higher total number of offspring weaned).

\textbf{Sample size}: we recorded the sample size for the relevant
statistical comparison (number of females, number of offspring, number
of matrilines etc.).

\textbf{Sampling variance}: we calculated the sampling variance of the
effect sizes based on the correlation coefficient r and the sample size,
using the formulas provided by Wilson (2019). The standard error, which
is alternatively used in some approaches, is the square root of the
sampling variance (Viechtbauer (2010)).

\textbf{Species identity}: we recorded the common name and the latin
species name as listed by the authors. We referred to the Mammal
Diversity Database (Burgin et al. (2018)) to resolve instances where
species attributions had been changed since the publication of the
original study.

\textbf{Study site}: we recorded the name of the study site as listed by
the authors in the method section. The focus of this variable is to
determine whether multiple observations are from the same species from
the same study population, and we accordingly assigned different names
for the study site label in case two or more different species had been
studied at the same site.

\textbf{Measure of reproductive success}: we recorded which aspect of
reproduction dominance rank was associated with. We classified
reproductive traits into six classes: - age at first reproduction
(includes age at first birth, age at first conception, age at first
menstrual cycle); - infant survival (includes rates of mortality of
offspring prior to their independence; proportion of pregnancies carried
to birth); - survival (includes rates of mortality of females per year,
age at death); - infant production (includes litter size, offspring
weight, litter mass, number of offspring per year, probability of birth
in a given year, number of surviving infants per year); - interbirth
interval (includes time between life births, number of cycles to
conception, number of litters per year); - lifetime reproductive success
(includes total number of offspring born or surviving to independence
for females who had been observed from first reproduction to death).

\textbf{Classification of rank}: we recorded the approach the authors
had used to assign dominance positions to individuals, distinguishing
between those based on aggressive/submissive interactions between pairs
of individuals and those based on other traits such as age, size, or
which female was the first to reproduce.

\textbf{Scoring of rank}: we recorded whether in the analyses
individuals were assigned a specific, continuous rank position or
whether individuals were classified into rank categories (dominant
versus subordinates, high- versus middle- versus low-ranking).

\textbf{Duration of study}: we recorded the number of years that authors
had observed the individuals (anything less than one year was assigned a
value of 1).

\textbf{Population type}: we recorded whether the population was
free-living, provisioned, or captive based on the authors descriptions.

\textbf{Social group size}: we recorded the average number of adult
females per group in the study population, based on the information
provided in the manuscripts. We relied on the definition of a social
group as used by the respective authors, which might include
associations of females in: singular-breeder cooperative groups (as in
wolves or meerkats); stable groups of multiple breeding females (as in
baboons or hyenas); or breeding associations defined by physical
proximity (as in bighorn sheep or antelopes). We will have a separate
coding of the social system (see below). Where available, we also coded
the average number of adult males associated with each group of females
to determine the sex ratio in social groups as a proxy for intersexual
conflict.

\textbf{Variables extracted from the broader literature for each
species/population:} This data will be added prior to the analyses,
depending on which specific prediction(s) we will test. For most of
these, we will extract information from the relevant papers or
publications reporting on the same population. For some of these, we
will use previously published species' averages, because records from
each population for each specific period during which the effect of
dominance rank on reproductive success were measured are unlikely to be
available for a large enough sample. We list the likely sources we will
use to obtain these data.

\textbf{Litter size}: the number of offspring per birth; data available
for each population, we will use the average as reported by the authors.

\textbf{Interbirth interval}: the time in months between consecutive
births; data available for a limited set of populations, we will use the
average as reported by the authors. Depending on the availability of
population specific data, we will alternatively use averages as reported
for each species (based on the data in Jones et al. (2009)).

\textbf{Cooperative breeding group}: whether social groups usually
contain a single breeding female and additional non-breeding adult
females that help to raise the offspring of the breeding female. Group
membership is usually closed and changes occur through birth and death
or fissioning of existing groups. This classification is in contrast to
plural breeding groups and breeding associations (see below); data
available for each population, we will use the description of the social
system in the population as reported by the authors.

\textbf{Plural breeding group}: whether social groups usually contain
multiple breeding females that remain together for extended periods of
time. It includes both groups in which females are philopatric or
disperse. Females form differentiated relationships with other group
members. This classification is in contrast to cooperative breeding
groups and breeding associations (see above/below); data available for
each population, we will use the description of the social system in the
population as reported by the authors.

\textbf{Breeding association}: whether social groups consist of multiple
breeding females that associate either in space or by mutual attraction.
Group membership is fluid and associations among individuals can rapidly
change. This classification is in contrast to cooperative breeding
groups and plural breeding groups (see above); data available for each
population, we will use the description of the social system in the
population as reported by the authors.

\textbf{Dominance system}: whether dominance rank of females appears to
depend primarily on (i) their age, (ii) their physical attributes such
as body size, (iii) support from their mother, or (iv) coalitionary
support from same-aged group members. Data available from a subset of
populations, to which we will add from primary reports of species-level
classifications from other populations assuming that this trait is
usually stable across populations within species.

\textbf{Philopatry}: whether females have the majority of their
offspring in the same social groups or in the same location in which
they have been born or whether females disperse to other groups or
locations to reproduce; data from species-level descriptions of female
behaviour (based on the data in Barsbai, Lukas, and Pondorfer (n.d.)).

\textbf{Monopolizable resources}: whether the gross dietary category of
a species is based on monopolizable resources (carnivory, frugivory), or
non-monopolizable resources (herbivory, or omnivory) (based on the data
in Wilman et al. (2014)).

\textbf{Environmental harshness}: whether the average climatic
conditions experienced by the species are characterized by cold
temperatures, low rainfall, and unpredictability (based on the data in
Botero et al. (2014)).

\textbf{Population density}: the average number of individuals per
square kilometer for the species (based on the data in Jones et al.
(2009)).

\textbf{Average and variance in relatedness among group females}: the
average and variance in relatedness measured using genetic approaches
among adult females within the same group as reported for this species;
data available from a subset of the populations.

\textbf{Coalition formation}: whether adult females form coalitions with
other female group members to support each other during within-group
aggressive interactions; data from species-level descriptions of female
behaviour (based on the data in Lukas and Clutton-Brock (2018)).

\hypertarget{phylogeny}{%
\paragraph{\texorpdfstring{\textbf{Phylogeny}}{Phylogeny}}\label{phylogeny}}

We will generate a single consensus phylogeny for the mammalian species
in our sample from the most recent complete mammalian time-calibrated
phylogeny (Upham, Esselstyn, and Jetz (2019)). We will download a
credible set of 1000 trees of mammalian phylogenetic history from
vertlife.org/phylosubsets/ and use TreeAnnotator (version 1.8.2 in
BEAST: Drummond et al. (2012)) to generate a maximum clade credibility
(MCC) tree (median node heights and a burn in of 250 trees). We will
trim the tree to match the species in our sample and convert branch
lengths using functions of the package ape (Paradis and Schliep (2019)).

~

\hypertarget{f.-analysis-plan}{%
\subsubsection{F. ANALYSIS PLAN}\label{f.-analysis-plan}}

We will perform all analyses in the statistical software R (R Software
Consortium 2019). We will build separate models for each prediction. To
assess the robustness of the findings and whether modeling decisions
might have an influence on our results, we will use a frequentist and a
Bayesian approach to build the statistical models. We will first
estimate all models using functions in the package metafor (Viechtbauer
(2010)). We will fit meta-analytic multilevel mixed-effects models with
moderators via linear models, including models that account for the
potential correlations among effect sizes due to shared phylogenetic
history among species (Nakagawa and Santos (2012)). Second, we will
estimate a subset of models using Bayesian approaches in the package
rethinking (McElreath (2020)). We will fit multilevel models that
include the sampling variance as measurement error (Kurz (2019)) and
shared phylogenetic history as a covariance matrix. Weakly regularizing
priors will be used for all parameters. The models will be implemented
in Stan. We will draw 8000 samples from four chains, checking that for
each the R-hat values will be less than 1.01. Visual inspection of trace
plots and rank histograms will be performed to ensure that they
indicated no evidence of divergent transitions or biased posterior
exploration. Posteriors from the model will be used to generate
estimates of the overall effect size and the influence of potential
moderators. We detail model construction in the following: we will first
assess whether species and population identity create dependencies
amongst the measured effect sizes. If so, we will include these factors
through covariance matrices reflecting the dependence across
measurements. We provide code showing the setup of the various models,
together with a simulated dataset with the same structure as the actual
data to
\href{https://github.com/dieterlukas/FemaleDominanceReproduction_MetaAnalysis}{assess
the code here}

\textbf{Publication bias}: We will plot all effect sizes in a
funnel-plot to perform a visual inspection of the range of effect sizes
at different sample sizes and to investigate whether there might be an
underrepresentation of studies with small effect sizes and small sample
sizes (Egger et al. (1997)). Given the diversity of studies in our
sample, we expect that the effect sizes will not represent a sample from
a single distribution: for example, studies of offspring mortality tend
to have larger sample sizes (because each mother can have multiple
offspring) and we predict different effect sizes for these studies.
Accordingly, we cannot perform tests to determine signs of publication
bias such as skewness or missing values for the whole sample.

\textbf{Overall effect}: We will construct a multilevel intercept-only
meta-analytic base model to test for a general effect of dominance rank
on reproductive success.

\textbf{Influence of locality/species}: To the base model, we will add
random effects to account for non-independence due to effect sizes
originating from within the same study, from studies performed on the
same population and on the same species.

\textbf{Influence of approach}: To the base model, we will add random
effects reflecting the differences in approaches across studies
(wild/captive; agonism/correlate; linear/categorical rank).

\textbf{Influence of measure of reproductive success}: To the base
model, we will add a predictor variable reflecting the six classes of
measures of reproductive success.

\textbf{Influence of phylogeny}: To the random effects model, we will
add a covariance structure to reflect potential similarities in effect
sizes arising from closely related species showing similar effects due
to their shared phylogenetic history.

\textbf{Influence of predictor variables}: Most models we construct will
be univariate, testing the influence of a single variable at a time to
assess support for the specific predictions. In case any of the previous
models suggests dependencies among the measured effect sizes, we will
add the predictor variables as moderators to models including covariance
matrices reflecting the dependencies (e.g.~if effect sizes are different
for different measures of reproductive success). For the variables
reflecting the social environment, we will build multivariate models to
assess potential pathways. To assess whether average relatedness
directly influences the effect of dominance on reproduction we will also
include the number of females per group; and to assess whether coalition
formation mediates how average relatedness shapes dominance effects, we
will build a model with these two factors included. For instances where
we might expect covariation among variables that are predicted to
influence the strength of the effect sizes in opposite ways, we will
build models that include both variables and their interaction. For the
predictions in which we expect different associationships in cooperative
breeders as compared to plural breeders, we will nest the influence of
the predictor variable within the classification of the social system in
a multilevel model. Studies performed on wild versus captive individuals
and using different measures of reproductive success might not only
differ in the overall strength of the effect of rank on reproductive
success, but also in how other variables influence this effect. We
therefore will build models in which both the intercept and the slopes
can vary according to whether studies were performed in the
wild/captivity and to how reproductive success was measured.

~

\hypertarget{g.-ethics}{%
\subsubsection{G. ETHICS}\label{g.-ethics}}

Our study relies on previously published data.

\hypertarget{h.-author-contributions}{%
\subsubsection{H. AUTHOR CONTRIBUTIONS}\label{h.-author-contributions}}

\textbf{Shivani:} Hypothesis development, data collection, data analysis
and interpretation, revising/editing.

\textbf{Huchard:} Hypothesis development, data analysis and
interpretation, write up, revising/editing.

\textbf{Lukas:} Hypothesis development, data collection, data analysis
and interpretation, write up, revising/editing, materials/funding.

\hypertarget{i.-funding}{%
\subsubsection{I. FUNDING}\label{i.-funding}}

Shivani received funding from the INSPIRE programme of the Department of
Science \& Technology of the Government of India. This research was
supported by the Department of Human Behavior, Ecology and Culture at
the Max Planck Institute for Evolutionary Anthropology.

\hypertarget{j.-conflict-of-interest-disclosure}{%
\subsubsection{J. CONFLICT OF INTEREST
DISCLOSURE}\label{j.-conflict-of-interest-disclosure}}

We, the authors, declare that we have no financial conflicts of interest
with the content of this article. Elise Huchard and Dieter Lukas are
Recommenders at PCI Ecology.

~

\hypertarget{k.-references}{%
\subsubsection*{\texorpdfstring{K.
\href{MyLibrary.bib}{REFERENCES}}{K. REFERENCES}}\label{k.-references}}
\addcontentsline{toc}{subsubsection}{K.
\href{MyLibrary.bib}{REFERENCES}}

\hypertarget{refs}{}
\leavevmode\hypertarget{ref-alexander1974evolution}{}%
Alexander, Richard D. 1974. ``The Evolution of Social Behavior.''
\emph{Annual Review of Ecology and Systematics} 5 (1). Annual Reviews
4139 El Camino Way, PO Box 10139, Palo Alto, CA 94303-0139, USA:
325--83.

\leavevmode\hypertarget{ref-barsbai2020similarity}{}%
Barsbai, Toman, Dieter Lukas, and Andreas Pondorfer. n.d. ``Local
Convergence of Behavior Across Species.'' \emph{OSF Preprints,
Https://Osf.io/U839m/}.

\leavevmode\hypertarget{ref-bercovitch1991social}{}%
Bercovitch, Fred B. 1991. ``Social Stratification, Social Strategies,
and Reproductive Success in Primates.'' \emph{Ethology and Sociobiology}
12 (4). Elsevier: 315--33.

\leavevmode\hypertarget{ref-botero2014environmental}{}%
Botero, Carlos A, Roi Dor, Christy M McCain, and Rebecca J Safran. 2014.
``Environmental Harshness Is Positively Correlated with Intraspecific
Divergence in Mammals and Birds.'' \emph{Molecular Ecology} 23 (2).
Wiley Online Library: 259--68.

\leavevmode\hypertarget{ref-brown2002reconsidering}{}%
Brown, Gillian R, and Joan B Silk. 2002. ``Reconsidering the Null
Hypothesis: Is Maternal Rank Associated with Birth Sex Ratios in Primate
Groups?'' \emph{Proceedings of the National Academy of Sciences} 99
(17). National Acad Sciences: 11252--5.

\leavevmode\hypertarget{ref-burgin2018many}{}%
Burgin, Connor J, Jocelyn P Colella, Philip L Kahn, and Nathan S Upham.
2018. ``How Many Species of Mammals Are There?'' \emph{Journal of
Mammalogy} 99 (1). Oxford University Press US: 1--14.

\leavevmode\hypertarget{ref-chamberlain2012does}{}%
Chamberlain, Scott A, Stephen M Hovick, Christopher J Dibble, Nick L
Rasmussen, Benjamin G Van Allen, Brian S Maitner, Jeffrey R Ahern, et
al. 2012. ``Does Phylogeny Matter? Assessing the Impact of Phylogenetic
Information in Ecological Meta-Analysis.'' \emph{Ecology Letters} 15
(6). Wiley Online Library: 627--36.

\leavevmode\hypertarget{ref-cheney2004factors}{}%
Cheney, Dorothy L, Robert M Seyfarth, Julia Fischer, J Beehner, T
Bergman, SE Johnson, Dawn M Kitchen, RA Palombit, D Rendall, and Joan B
Silk. 2004. ``Factors Affecting Reproduction and Mortality Among Baboons
in the Okavango Delta, Botswana.'' \emph{International Journal of
Primatology} 25 (2). Springer: 401--28.

\leavevmode\hypertarget{ref-clutton2013social}{}%
Clutton-Brock, T, and E Huchard. 2013. ``Social Competition and Its
Consequences in Female Mammals.'' \emph{Journal of Zoology} 289 (3).
Wiley Online Library: 151--71.

\leavevmode\hypertarget{ref-clutton2010adaptive}{}%
Clutton-Brock, Tim H, Sarah J Hodge, Tom P Flower, Goran F Spong, and
Andrew J Young. 2010. ``Adaptive Suppression of Subordinate Reproduction
in Cooperative Mammals.'' \emph{The American Naturalist} 176 (5). The
University of Chicago Press: 664--73.

\leavevmode\hypertarget{ref-digby2006role}{}%
Digby, Leslie J, Stephen F Ferrari, and Wendy Saltzman. 2006. ``The Role
of Competition in Cooperatively Breeding Species.'' \emph{Primates in
Perspective. Oxford University Press, New York}. Citeseer, 85--106.

\leavevmode\hypertarget{ref-drummond2012bayesian}{}%
Drummond, Alexei J, Marc A Suchard, Dong Xie, and Andrew Rambaut. 2012.
``Bayesian Phylogenetics with Beauti and the Beast 1.7.''
\emph{Molecular Biology and Evolution} 29 (8). Oxford University Press:
1969--73.

\leavevmode\hypertarget{ref-egger1997bias}{}%
Egger, Matthias, George Davey Smith, Martin Schneider, and Christoph
Minder. 1997. ``Bias in Meta-Analysis Detected by a Simple, Graphical
Test.'' \emph{Bmj} 315 (7109). British Medical Journal Publishing Group:
629--34.

\leavevmode\hypertarget{ref-ellis1995dominance}{}%
Ellis, Lee. 1995. ``Dominance and Reproductive Success Among Nonhuman
Animals: A Cross-Species Comparison.'' \emph{Ethology and Sociobiology}
16 (4). Elsevier: 257--333.

\leavevmode\hypertarget{ref-fedigan1983dominance}{}%
Fedigan, Linda Marie. 1983. ``Dominance and Reproductive Success in
Primates.'' \emph{American Journal of Physical Anthropology} 26 (S1).
Wiley Online Library: 91--129.

\leavevmode\hypertarget{ref-fedigan2013sexual}{}%
Fedigan, Linda Marie, and Katharine M Jack. 2013. ``Sexual Conflict in
White-Faced Capuchins.'' \emph{Evolution's Empress, Eds Fisher ML,
Garcia JR (Oxford Univ Press, New York)}, 281--303.

\leavevmode\hypertarget{ref-fortunato2019lineal}{}%
Fortunato, Laura. 2019. ``Lineal Kinship Organization in Cross-Specific
Perspective.'' \emph{Philosophical Transactions of the Royal Society B}
374 (1780). The Royal Society: 20190005.

\leavevmode\hypertarget{ref-frank1986social}{}%
Frank, Laurence G. 1986. ``Social Organization of the Spotted Hyaena
Crocuta Crocuta. II. Dominance and Reproduction.'' \emph{Animal
Behaviour} 34 (5). Elsevier: 1510--27.

\leavevmode\hypertarget{ref-giles2015dominance}{}%
Giles, Sarah L, Christine J Nicol, Patricia A Harris, and Sean A Rands.
2015. ``Dominance Rank Is Associated with Body Condition in
Outdoor-Living Domestic Horses (Equus Caballus).'' \emph{Applied Animal
Behaviour Science} 166. Elsevier: 71--79.

\leavevmode\hypertarget{ref-holekamp1991dominance}{}%
Holekamp, Kay E, and Laura Smale. 1991. ``Dominance Acquisition During
Mammalian Social Development: The `Inheritance' of Maternal Rank.''
\emph{American Zoologist} 31 (2). Oxford University Press UK: 306--17.

\leavevmode\hypertarget{ref-huchard2016competitive}{}%
Huchard, Elise, Sinead English, Matt BV Bell, Nathan Thavarajah, and Tim
Clutton-Brock. 2016. ``Competitive Growth in a Cooperative Mammal.''
\emph{Nature} 533 (7604). Nature Publishing Group: 532--34.

\leavevmode\hypertarget{ref-jones2009pantheria}{}%
Jones, Kate E, Jon Bielby, Marcel Cardillo, Susanne A Fritz, Justin
O'Dell, C David L Orme, Kamran Safi, et al. 2009. ``PanTHERIA: A
Species-Level Database of Life History, Ecology, and Geography of Extant
and Recently Extinct Mammals: Ecological Archives E090-184.''
\emph{Ecology} 90 (9). Wiley Online Library: 2648--8.

\leavevmode\hypertarget{ref-kurz2019rethinking}{}%
Kurz, Solomon. 2019. \emph{Statistical Rethinking with Brms, Ggplot2,
and the Tidyverse}. available at:
https://solomonkurz.netlify.com/post/bayesian-meta-analysis/.

\leavevmode\hypertarget{ref-lajeunesse2013recovering}{}%
Lajeunesse, Marc J, J Koricheva, J Gurevitch, and K Mengersen. 2013.
``Recovering Missing or Partial Data from Studies: A Survey of
Conversions and Imputations for Meta-Analysis.'' \emph{Handbook of
Meta-Analysis in Ecology and Evolution}. Princeton University Press:
Princeton, New Jersey, 195--206.

\leavevmode\hypertarget{ref-lakens2013calculating}{}%
Lakens, Daniël. 2013. ``Calculating and Reporting Effect Sizes to
Facilitate Cumulative Science: A Practical Primer for T-Tests and
Anovas.'' \emph{Frontiers in Psychology} 4. Frontiers: 863.

\leavevmode\hypertarget{ref-lukas2017climate}{}%
Lukas, Dieter, and Tim Clutton-Brock. 2017. ``Climate and the
Distribution of Cooperative Breeding in Mammals.'' \emph{Royal Society
Open Science} 4 (1). The Royal Society Publishing: 160897.

\leavevmode\hypertarget{ref-lukas2018social}{}%
---------. 2018. ``Social Complexity and Kinship in Animal Societies.''
\emph{Ecology Letters} 21 (8). Wiley Online Library: 1129--34.

\leavevmode\hypertarget{ref-lukas2019evolution}{}%
Lukas, Dieter, and Elise Huchard. 2019. ``The Evolution of Infanticide
by Females in Mammals.'' \emph{Philosophical Transactions of the Royal
Society B} 374 (1780). The Royal Society: 20180075.

\leavevmode\hypertarget{ref-lukas2005extent}{}%
Lukas, Dieter, Vernon Reynolds, Christophe Boesch, and Linda Vigilant.
2005. ``To What Extent Does Living in a Group Mean Living with Kin?''
\emph{Molecular Ecology} 14 (7). Wiley Online Library: 2181--96.

\leavevmode\hypertarget{ref-majolo2012fitness}{}%
Majolo, Bonaventura, Julia Lehmann, Aurora de Bortoli Vizioli, and
Gabriele Schino. 2012. ``Fitness-Related Benefits of Dominance in
Primates.'' \emph{American Journal of Physical Anthropology} 147 (4).
Wiley Online Library: 652--60.

\leavevmode\hypertarget{ref-mcelreath2020statistical}{}%
McElreath, Richard. 2020. \emph{Statistical Rethinking: A Bayesian
Course with Examples in R and Stan}. CRC press.

\leavevmode\hypertarget{ref-nakagawa2012methodological}{}%
Nakagawa, Shinichi, and Eduardo SA Santos. 2012. ``Methodological Issues
and Advances in Biological Meta-Analysis.'' \emph{Evolutionary Ecology}
26 (5). Springer: 1253--74.

\leavevmode\hypertarget{ref-pandit2003model}{}%
Pandit, Sagar A, and Carel P van Schaik. 2003. ``A Model for Leveling
Coalitions Among Primate Males: Toward a Theory of Egalitarianism.''
\emph{Behavioral Ecology and Sociobiology} 55 (2). Springer: 161--68.

\leavevmode\hypertarget{ref-paradis2019ape}{}%
Paradis, Emmanuel, and Klaus Schliep. 2019. ``Ape 5.0: An Environment
for Modern Phylogenetics and Evolutionary Analyses in R.''
\emph{Bioinformatics} 35 (3). Oxford University Press: 526--28.

\leavevmode\hypertarget{ref-pusey2012magnitude}{}%
Pusey, Anne. 2012. ``Magnitude and Sources of Variation in Female
Reproductive Performance.'' \emph{The Evolution of Primate Societies}.
University of Chicago Press Chicago, IL, 343--66.

\leavevmode\hypertarget{ref-solomon1997cooperative}{}%
Solomon, Nancy G, Jeffrey A French, and others. 1997. \emph{Cooperative
Breeding in Mammals}. Cambridge University Press.

\leavevmode\hypertarget{ref-stockley2011female}{}%
Stockley, Paula, and Jakob Bro-Jørgensen. 2011. ``Female Competition and
Its Evolutionary Consequences in Mammals.'' \emph{Biological Reviews} 86
(2). Wiley Online Library: 341--66.

\leavevmode\hypertarget{ref-thouless1986conflict}{}%
Thouless, CR, and FE Guinness. 1986. ``Conflict Between Red Deer Hinds:
The Winner Always Wins.'' \emph{Animal Behaviour} 34 (4). Elsevier:
1166--71.

\leavevmode\hypertarget{ref-upham2019inferring}{}%
Upham, Nathan S, Jacob A Esselstyn, and Walter Jetz. 2019. ``Inferring
the Mammal Tree: Species-Level Sets of Phylogenies for Questions in
Ecology, Evolution, and Conservation.'' \emph{PLoS Biology} 17 (12).
Public Library of Science.

\leavevmode\hypertarget{ref-van1988scramble}{}%
Van Noordwijk, Maria A, and Carel P Van Schaik. 1988. ``Scramble and
Contest in Feeding Competition Among Female Long-Tailed Macaques (Macaca
Fascicularis).'' \emph{Behaviour} 105 (1-2). Brill: 77--98.

\leavevmode\hypertarget{ref-vehrencamp1983model}{}%
Vehrencamp, Sandra L. 1983. ``A Model for the Evolution of Despotic
Versus Egalitarian Societies.'' \emph{Animal Behaviour} 31 (3).
Elsevier: 667--82.

\leavevmode\hypertarget{ref-viechtbauer2010conducting}{}%
Viechtbauer, Wolfgang. 2010. ``Conducting Meta-Analyses in R with the
Metafor Package.'' \emph{Journal of Statistical Software} 36 (3). UCLA
Statistics: 1--48.

\leavevmode\hypertarget{ref-ward2016sociality}{}%
Ward, Ashley, and Mike Webster. 2016. ``Sociality: The Behaviour of
Group-Living Animals.'' Springer.

\leavevmode\hypertarget{ref-west1967foundress}{}%
West, Mary Jane. 1967. ``Foundress Associations in Polistine Wasps:
Dominance Hierarchies and the Evolution of Social Behavior.''
\emph{Science} 157 (3796). American Association for the Advancement of
Science: 1584--5.

\leavevmode\hypertarget{ref-williams1966adaptation}{}%
Williams, George C. 1966. \emph{Adaptation and Natural Selection: A
Critique of Some Current Evolutionary Thought}. Vol. 833082108.
Princeton science library OCLC.

\leavevmode\hypertarget{ref-wilman2014eltontraits}{}%
Wilman, Hamish, Jonathan Belmaker, Jennifer Simpson, Carolina De La
Rosa, Marcelo M Rivadeneira, and Walter Jetz. 2014. ``EltonTraits 1.0:
Species-Level Foraging Attributes of the World's Birds and Mammals:
Ecological Archives E095-178.'' \emph{Ecology} 95 (7). Wiley Online
Library: 2027--7.

\leavevmode\hypertarget{ref-wilson2019calculator}{}%
Wilson, D. B. 2019. \emph{Ractical Meta-Analysis Effect Size Calculator
{[}Online Calculator{]}.} retrieved from:
hhttps:/www.campbellcollaboration.org/research-resources/research-for-resources/effect-size-calculator.html.

\end{document}
